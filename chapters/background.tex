\section{Introduction}
\section{Existing Solutions}
\subsection*{Fandom Wiki}
The Binding of Isaac has two wiki sites hosted on the Fandom Wiki platform; one for the original flash 
game\cite{BindingIsaacWiki}, and one for the modern version, commonly referred to as 'Rebirth'\cite{BindingIsaacRebirth}.
For the purposes of this project we will only be considering the modern version as it is widely considered the 'goto' 
version within the game's community.\par The website is contains comprehensive information on all aspects of the game, 
and it is continualy updated by the community. Users can navigate the site using either predefined categories or a 
powerful search tool. \par
Advantages
\begin{itemize}
    \item Contains information on all aspects of the game
    \item Actively maintained by the community
    \item Usful search functionality
\end{itemize}
Disadvantages
\begin{itemize}
    \item So much information can make it hard to find what is relevant
    \item Unable to search for interactions, have to go via each item 
\end{itemize}
\subsection*{Platinmum God}
Platinmum God is a self-described `Isaac Cheat Sheet'\cite{IsaacCheatSheet} and it contains item and key mechanic 
information for all versions of the game. The site is maintained by one person, and it claims to be more accurate than 
the community wiki as it update is `tested thoroughly in the game using Cheat Engine'\cite{FrequentlyAskedQuestions}. 
The information is split into pages based on the version of the game; users can navigate this using the item icons which
 are arrayed on the page, or by using the search functionality. The search tool has some supported keywords, but will 
still usually require entering an exact match to an entry in the data. For certain versions of the game there is also a 
synergy finder tool which lets the user enter two items to see how they interact. However, this is limited to older 
versions of the game and only a small set of the items are actually included in the tool. \par
Advantages
\begin{itemize}
    \item Information is more reliable than the community wiki
    \item Easier to reference quickly due to there being less information
\end{itemize}
Disadvantages
\begin{itemize}
    \item Only one maintainer can mean long update times
    \item Only contains basic information about each item
    \item Limited or no synergy information for most items
    \item Harder to find items without knowing the name or what the item looks like
\end{itemize}
\section{Technology Review}
\subsection{Client Side Framework}
\subsubsection*{Angular}
`Angular is an application-design framework and development platform for creating efficient and sophisticated 
single-page apps.'\cite{AngularIntroductionAngular}
% Made by google
% free open source license
% can be used as a full framework but not interested in that
% Typescript
% Very popular and backed by google so lots of documentation
\subsubsection*{React}
`React is a declarative, efficient, and flexible JavaScript library for building user interfaces. It lets you compose 
complex UIs from small and isolated pieces of code called “components”.'\cite{TutorialIntroReact}
% maintained by Meta
% open source
% library not a framework
% JSX
\subsubsection*{Vue}
`An approachable, performant and versatile framework for building web user interfaces.'\cite{VueJsProgressive}
% Not as popular and so less documentation and community support
% framework
% very fast and designed to be flexible
\subsubsection*{Conclusion}
% Chose Angular because:
% - Used on placement
\subsection{Server Side Framework}
% Only looked at python frameworks because of previous experience and neomodel
\subsubsection*{Django}
`Django is a high-level Python web framework that encourages rapid development and clean, pragmatic design.'\cite{Django}
% can be used as a full framework but not interested
% Pretty much everythin you need comes as default (but also a lot that you dont need)
\subsubsection*{Flask}
`Flask is a lightweight WSGI web application framework.'\cite{ronacherFlaskSimpleFramework}
% No database abstraction layer
% Base install is quite minimal so needs extensions
\subsubsection*{Conclusion}
% Chose Django because:
% - Used it more on placement
\subsection{Database}
\subsubsection*{Neo4j}
`Neo4j is an open-source, NoSQL, native graph database that provides an ACID-compliant transactional backend for your 
applications'\cite{WhatGraphDatabaseb}
% - Aura platform for hosting the database remotely
%   - Good for working on different machines
%   - Removes dependency on local resources while developing
%   - Bloom is good for viewing data while developing
%   - Data importer
% - Has lots of documentation and tutorials
% - Free Aura instance
\subsubsection*{Amazon Neptune}
`Amazon Neptune is a purpose-built, high-performance graph database engine optimized for storing billions of 
relationships and querying the graph with milliseconds latency.'\cite{WhatGraphDatabase}
% Not completely free
\subsubsection*{Conclusion}
% Chose Neo4j because:
% Free
% more documentation
The decision was made to use Neo4j. This is primarily because the Aura platfrom provides a permanent free database 
instance which has ample resources for this project. There also exists a Python libraries, neomodel and django-neomodel,
 for easily integrating database access in Django.

% \subsection{Hosting}
% \subsubsection*{AWS}
% \subsubsection*{Azure}
% \subsubsection*{Google Cloud}
% \subsection{CI/CD}
% \subsubsection*{GitHub}
% \subsubsection*{CircleCI}
% \subsubsection*{Jenkins}
\section{Conclusion}